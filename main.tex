\documentclass[a4paper]{report}
\linespread{1.5} % ajouter un espacement entre les lignes de tout le document 
%====================== PACKAGES ======================
%\documentclass{article}
\usepackage{xcolor}
\usepackage{listings}
\definecolor{mGreen}{rgb}{0,0.6,0}
\definecolor{mGray}{rgb}{0.5,0.5,0.5}
\definecolor{mPurple}{rgb}{0.58,0,0.82}
\definecolor{backgroundColour}{rgb}{0.95,0.95,0.92}

\lstdefinestyle{CStyle}{
    backgroundcolor=\color{backgroundColour},   
    commentstyle=\color{mGreen},
    keywordstyle=\color{magenta},
    numberstyle=\tiny\color{mGray},
    stringstyle=\color{mPurple},
    basicstyle=\footnotesize,
    breakatwhitespace=false,         
    breaklines=true,                 
    captionpos=b,                    
    keepspaces=true,                 
    numbers=left,                    
    numbersep=5pt,                  
    showspaces=false,                
    showstringspaces=false,
    showtabs=false,                  
    tabsize=2,
    language=java
}






\usepackage{multirow}
\usepackage{algorithm}
\usepackage{algorithmic}
\usepackage[french]{babel}
\usepackage[utf8x]{inputenc}
%pour gérer les positionnement d'images
\usepackage{float}
\usepackage{amsmath}
\usepackage{graphicx}
\usepackage[colorinlistoftodos]{todonotes}
\usepackage{url}
%pour les informations sur un document compilé en PDF et les liens externes / internes
\usepackage{hyperref}
%pour la mise en page des tableaux
\usepackage{array}
\usepackage{tabularx}
%pour utiliser \floatbarrier
%\usepackage{placeins}
%\usepackage{floatrow}
%espacement entre les lignes
\usepackage{setspace}
%modifier la mise en page de l'abstract
\usepackage{abstract}
%police et mise en page (marges) du document
\usepackage[T1]{fontenc}
\usepackage[top=2cm, bottom=2cm, left=2cm, right=2cm]{geometry}
%Pour les galerie d'images
\usepackage{subfig}

\usepackage{longtable}
\usepackage{svg}
\svgpath{{../imgs/}}

%============= INFORMATION ET REGLES =============

%rajouter les numérotation pour les \paragraphe et \subparagraphe
\setcounter{secnumdepth}{4}
\setcounter{tocdepth}{4}

\hypersetup{							% Information sur le document
pdfauthor = {EL MRABTI Hamza},			% Auteurs
pdftitle = {Système de santé},			% Titre du document
pdfsubject = {Système de santé},		% Sujet
pdfkeywords = {Tag1, Tag2, Tag3, ...},	% Mots-clefs
pdfstartview={FitH}}					% ajuste la page à la largueur de l'écran
%pdfcreator = {MikTeX},% Logiciel qui a crée le document
%pdfproducer = {}} % Société avec produit le logiciel

%======================== DEBUT DU DOCUMENT ========================

\begin{document}

%régler l'espacement entre les lignes
\newcommand{\HRule}{\rule{\linewidth}{0.5mm}}

%page de garde
\begin{titlepage}
\begin{center}

\begin{minipage}{0.45\textwidth} 
\begin{flushleft}
    \includegraphics[scale = 0.4]{Images/ensias.png}
\end{flushleft}
\end{minipage}
\begin{minipage}{0.48\textwidth} 
\begin{flushright}
    \includegraphics[scale = 0.5]{Images/nowedge.png}
\end{flushright}
\end{minipage}
\\[1.9cm] 
 


\textsc{\Large Ecole Nationale Supérieure d’Informatique et d’Analyse des Systèmes - RABAT }\\[2cm]

\textsc{\Large Rapport de Stage de la Première Année
}\\[0.5cm]

% Title
\HRule \\[0.4cm]

{\huge \bfseries   
Conception et développement du back-end de Serious Games en langage Python
 \\[0.4cm] }
\HRule \\
[2cm]
% Author and supervisor
\begin{minipage}{0.4\textwidth}
\begin{flushleft} \large
\emph{Réalisé par :}\\
Hamza \textsc{El Mrabti}\\ [1cm]

\emph{Filière :} \\
\textsc{Génie Logiciel }\\



\end{flushleft}
\end{minipage}
\begin{minipage}{0.4\textwidth}
\begin{flushright} \large
\emph{Encadré par:} \\
Dr. Reda \textsc{Ahroum}\\[1cm]
\emph{Memebres du jury:} \\
Pr. Widad \textsc{Ettazi}\\
Pr. Salah \textsc{Baina}\\[1cm]


\end{flushright}
\end{minipage}

\vfill

% Bottom of the page
{\large \ Année Scolaire 2022/2023}

\end{center}
\end{titlepage}

%page blanche 
\newpage
~
%ne pas numéroter cette page
\thispagestyle{empty}
%\newpage  ==> pour créer une nouvelle page

%\thispagestyle{empty}
\chapter*{Dédicaces}

\vspace*{\stretch{0.5}}
\begin{flushright}
\emph{\`A ma famille, à mes amis, à tous ceux qui me sont chèrs } \\
\textsc{Hamza El Mrabti}\\
\end{flushright}

\vspace*{\stretch{0.2}}


\vspace*{\stretch{1}}




\thispagestyle{empty}

\chapter*{Remerciements}



\begin{center}
``La reconnaissance est la plus belle fleur qui jaillit de l'âme "
\end{center}
\begin{center} $\sim$ Henry Beecher . \end{center}
Je voudrais tout d'abord adresser toute ma gratitude à 
M. Reda Ahroum, mon encadrant, pour sa patience, sa disponibilité, sa rigueur scientifique, ses qualités humaines
et surtout ses judicieux conseils, qui ont contribué à alimenter ma réfexion.

Ensuite J'exprimer ma profonde reconnaissance aux membres du jury, notament Mme. Widad Ettazi et M. Salah Baina, pour le temps précieux qu'ils accordent en contribuant au bon déroulement du module.

Enfin je tiens à remercier chaleureusement ma famille, mes amis et toutes les personnes qui ont contribué énormément à la réalisation de ce projet et qui m'ont aidés lors de la rédaction de ce rapport, aussi bien pour leurs encouragements et soutiens dans les moments forts.


\thispagestyle{empty}
\chapter*{Résumé}

Le présent rapport porte sur le projet de fin du deuxième années du cycle d'ingénieur, filière génie logiciel à l'Ecole Supérieure d'Informatique et d'Analyse des Systèmes 

Le sujet de ce projet consistait à la gestion des données d'un patient dans un système multi-centres.

 Notre travail s’est fait en plusieurs phases:
 Une phase d’analyse et de conception et une phase de développement.
 Grâce à l’aide de notre encadrant, nous avons appris, durant ces derniers mois, à utiliser
différents outils et découvrir plusieurs concepts qui nous ont aidés à développer cette application.
 Nous vous présenterons donc tout au long de ce rapport, les étapes que nous avons
suivis ainsi que les outils que nous avons utilisés pour réaliser notre projet.

\textbf{Mot clès} : Système de santé , Données des patients , Dossier médical , e-Santé 

\thispagestyle{empty}
\chapter*{Abstract}


This report concerns the internship at the end of the first year of the engineering cycle, software engineering sector
at the Higher School of Computer Science and Systems Analysis
The subject of this project was the migration of serious games built by the company to a new
LMS (learning management system) and then perform unit tests on the LMS
Our work was done on two pillars: Migrating serious games to the new LMS and
perform unit tests on all the features of the latter. Thanks to the help of my supervisor, I learned,
over the past few months, using different tools and discovering several concepts that have helped us achieve
to the requested objectives. We will therefore present to you throughout this report, the steps we have followed
as well as the tools we used to carry out our project.

\textbf{Keywords} : Serious games , e-learning,  Onboarding , Migration, LMS, Tests

\thispagestyle{empty}
\listoffigures
\thispagestyle{empty}
\listoftables
\thispagestyle{empty}
\tableofcontents
\thispagestyle{empty}


\setcounter{page}{0}


\chapter*{Introduction générale}



La digitalisation de la santé est un processus lancé il y a quelques années à présent et qui commence seulement à prendre des formes concrètes dans la vie des patients, des médecins, des assureurs, des mutuelles, des organismes gouvernementaux et des laboratoires de santé surtout après la pandémie de la covide 19. Cet élan de digitalisation des entreprises s'inscrit dans un schéma global qui touche tous les secteurs de l'économie.

Cette digitalisation passe toujours par les données des patients appartenants au système de santé en questions, ce qui pose quelques problèmes lorsqu'on essaye de suivre les dossiers médicaux des patients sachant que leurs données sont stockées sur plusieurs bases de données, d'où vient l'intérêt de notre projet qui vise à centraliser ce traitement de gestion des données des patients, afin d'offrir aux proffessionnels de santé des interfaces unifiés sur les dossiers de leurs patients.



%ne pas numéroter le sommaire

% \newpage

%espacement entre les lignes d'un tableau
\renewcommand{\arraystretch}{1.5}

%====================== INCLUSION DES PARTIES ======================

~
%recommencer la numérotation des pages à "1"
\newpage

\chapter{Présentation du projet}

\section{Introduction}

Dans ce chapitre nous allons mettre le sujet dans son cadre général,le cahier de charge  et l’objectif de ce projet, ainsi pour obtenir une idée sur ce que va réaliser le système en termes de métier (comportement du système). Par la suite nous aborderons la méthode adoptée pour réaliser ce
projet.


\section{Problématique}


Face aux difficultés multiples que rencontre le personnel de la santé publique dans la gestion
 et l’exploitation des dossiers médicaux des patients, face aux flux de patients et compte tenu de l’importance que représente un dossier médical, l’informatisation de ce dossier présente un
enjeu stratégique majeur.
 Elle aura beaucoup de retombées sur la qualité des soins aux seins des établissements de santé d’où l’utilité et la nécessité d’un système capable de centraliser et capitaliser l’ensemble des informations générées par le patient.
 
 
\section{Description du projet}

Pour parer à ces difficultés, un système souple et sécurisé est nécessaire.
 C’est dans ce cadre que s’inscrit notre projet de fin d’année réalisé .Il s’agit d’une
Implémentation et conception d’une application web qui permet la sauvegarde, la gestion
et la manipulation des données médicaux relatives aux traitements subis par un patient.



\section{Objectifs}


L’objectif du projet est de concevoir un application web qui consiste à mémoriser
pour chaque patient, non seulement les informations administratives (âge, sexe, adresse,
contact,…), mais également des informations médicales (le diagnostic,les ordonances,
les comptes-rendus, les traitements administrés, les analyses et les résultats ,..) et tout
autre type d’actes médicaux … puisqu'il entend améliorer la saisie, la sauvegarde
et la communication des informations de santé, tout en respectant les droits du patient bien sûr , afin de garantir une gestion souple du dossier médical de ce patient.

 L’objectif sera également d’avoir une approche originale et différente par rapport au
 sujet et d’utiliser au mieux les outils dont nous disposons pour mettre au point ce projet.
La démarche que nous avons suivie pour mettre au point cette application est simple.
 Nous avons divisé notre travail en deux différentes phases:
 \begin{itemize}
 \item Une phase où nous avons analysé le sujet, décider une approche crédible pour modéliser
l’application.
 \item une phase de développement ou nous avons commencé à mettre au point l’application en créant la base de données et le site web en question.
 \end{itemize}
 

\section{Cahier de charge}

\subsection{Introduction}

Il s’agit de la conception d’un système qui gère les données des patients dans un système de santé multi-centres, il consiste à mémoriser pour chaque patient, non seulement les informations administratives (nom, âge, sexe ...), mais également des informations concernant l’historique médical du patient, ce qui aide les médecins consultés à prendre des décisions en se basant sur des données exactes et pertinentes, en offrant des interfaces personnalisées selon le profil du demandeur de l’information, et accessible à tout moment.\\
Pour cela on propose les utilisateurs suivants : 

\begin{itemize}
\item Patient
\item Médecin
\item Radiologue
\item Admin
\end{itemize}
 





\section{Conclusion}
Dans ce chapitre nous avons pu tracer le cadre générale du projet, pour préparer la surface pour l'analyse et la conception abordés dans le chapitre suivant.





\chapter{Analyse et conception}



\section{Introduction}

Ce chapitre abordera une analyse du sujet pour présenter une approche originale et une modélisation adéquate pour l’application


\subsection{Analyse du sujet}




\section{Conclusion} 

Aprés terminer cette étape d'analyse et conception nous avons une vision claire sur la réalisation qui est l'étape suivante qui vient aprés le choix des outils technologique.




\chapter{Réalisation et résultats}

\section{Introduction}

Après avoir achevé l’étape d’analyse et conception de l’application, on va entamer dans ce chapitre la partie réalisation et implémentation dans laquelle on va s’assurer que le système est prêt pour être exploité par les utilisateurs finaux. Premièrement on choisit les technologies et puis on procède à la réalisation.


\section{Choix des technologies}

\subsection{UML}

Le Langage de Modélisation Unifié, de l'anglais Unified Modeling Language (UML), est un langage de modélisation graphique à base de pictogrammes conçu comme une méthode normalisée de visualisation dans les domaines du développement logiciel et en conception orientée objet.

\begin{figure}[!h]
\begin{center}
\includegraphics[height=8cm]{UML.svg.png}
\end{center}
\caption{UML}
\end{figure}


\subsection{Next js}

Next.js est un cadriciel (framework) gratuit et open source s'appuyant sur la librairie JavaScript React et sur la technologie Node.js.

Le framework permet de créer des applications web universelles ou parfois appelées isomorphiques, signifiant que le code source est partagé entre le client et le serveur, de même que le font ses concurrents (Gatsby, Nuxt.js, Blitz).

\begin{figure}[!h]
\begin{center}
\includegraphics[height=6cm]{Nextjs.svg.png}
\end{center}
\caption{NEXT.JS}
\end{figure}

\subsection{Postgresql}

PostgreSQL est un système de gestion de base de données relationnelle orienté objet puissant et open source qui est capable de prendre en charge en toute sécurité les charges de travail de données les plus complexes. Alors que MySQL donne la priorité à l'évolutivité et aux performances, Postgres donne la priorité à la conformité et à l'extensibilité SQL.

\begin{figure}[!h]
\begin{center}
\includegraphics[height=8cm]{Postgresql.svg.png}
\end{center}
\caption{Postgresql}
\end{figure}



\subsection{Cassandra}

Apache Cassandra est un système de gestion de base de données (SGBD) de type NoSQL conçu pour gérer des quantités massives de données sur un grand nombre de serveurs, assurant une haute disponibilité en éliminant les points de défaillance unique. Il permet une répartition robuste sur plusieurs centres de données 3, avec une réplication asynchrone sans nœud maître et une faible latence pour les opérations de tous les clients.

\begin{figure}[!h]
\begin{center}
\includegraphics[height=6cm]{Cassandra.svg.png}
\end{center}
\caption{Cassandra}
\end{figure}


\subsection{Tailwind CSS}

Tailwind CSS est un framework utility-fist CSS (feuilles de style en cascade) avec des classes prédéfinies que vous pouvez utiliser pour construire et concevoir des pages web directement dans votre balisage. Il vous permet d’écrire du CSS dans votre HTML sous la forme de classes prédéfinies.

\begin{figure}[!h]
\begin{center}
\includegraphics[height=5cm]{tailwindcss.png}
\end{center}
\caption{Tailwind CSS}
\end{figure}




\subsection{Latex}

LaTeX (dont le logo est \LaTeX) est un langage et un système de composition de documents. Il s'agit d'une collection de macro-commandes destinées à faciliter l'utilisation du « processeur de texte » TeX de Donald Knuth.

LaTeX permet de rédiger des documents dont la mise en page est réalisée automatiquement en se conformant du mieux possible à des normes typographiques. Une fonctionnalité distinctive de LaTeX est son mode mathématique, qui permet de composer des formules complexes.

\begin{figure}[!h]
\begin{center}
\includegraphics[height=6cm]{LaTeX.svg.png}
\end{center}
\caption{\LaTeX}
\end{figure}

\subsection{Visual Studio Code}

Visual Studio Code est un éditeur de code extensible développé par Microsoft pour Windows, Linux et macOS.

Les fonctionnalités incluent la prise en charge du débogage, la mise en évidence de la syntaxe, la complétion intelligente du code, les snippets, la refactorisation du code et Git intégré. Les utilisateurs peuvent modifier le thème, les raccourcis clavier, les préférences et installer des extensions qui ajoutent des fonctionnalités supplémentaires.

\begin{figure}[!h]
\begin{center}
\includegraphics[height=4cm]{vscode.svg.png}
\end{center}
\caption{vscode}
\end{figure}


\subsection{Strapi}

Strapi est un CMS (Content Management System) en open source qui aide les entreprises à concevoir des architectures personnalisées en vue d'assister les développeurs, responsables et équipes dans la création et la gestion de contenus pour les projets d'applications mobiles et de sites web. La plateforme prend plusieurs bases de données en charge (notamment MongoDB, MySQL, SQLite et Postgres) afin de favoriser l'échange d'informations entre différentes sources de données.

\begin{figure}[!h]
\begin{center}
\includegraphics[height=6cm]{strapi.png}
\end{center}
\caption{Strapi}
\end{figure}


\subsection{Vercel}

Vercel (anciennement connu sous le nom de ZEIT) est une plate-forme cloud qui permet aux développeurs d'héberger des sites Web et des services Web qui se déploient instantanément, évoluent automatiquement et ne nécessitent aucune supervision.

\begin{figure}[!h]
\begin{center}
\includegraphics[height=6cm]{vercel.png}
\end{center}
\caption{Vercel}
\end{figure}


\subsection{Heroku}

Heroku est une plate-forme cloud en tant que service (PaaS) prenant en charge plusieurs langages de programmation. L'une des premières plates-formes cloud, Heroku est en développement depuis juin 2007, lorsqu'elle ne prenait en charge que le langage de programmation Ruby, mais prend désormais en charge Java, Node.js, Scala, Clojure, Python, PHP et Go.[3] Pour cette raison, Heroku est considéré comme une plate-forme polyglotte car il dispose de fonctionnalités permettant à un développeur de créer, d'exécuter et de mettre à l'échelle des applications de la même manière dans la plupart des langages.


\begin{figure}[!h]
\begin{center}
\includegraphics[height=3cm]{heroku.svg.png}
\end{center}
\caption{Heroku}
\end{figure}


\section{Interfaces graphiques}

\subsection{Page de login}

Page de login pour s'authentifier comme utilisateur du système
\begin{figure}[!h]
\begin{center}
\includegraphics[height=10cm]{log.png}
\end{center}
\caption{Page de login }
\end{figure}


\subsection{Profile du patient}

le profile du patient contient ses informations personnelles.

\newpage


\begin{figure}[!h]
\centering
\begin{center}
\includegraphics[height=10cm,width=18cm]{profile.jpeg}
\end{center}
\caption{Profile du patient}
\end{figure}


\subsection{Dossier medical du patient}

Le dossier médical du patient contient les dernières actions faites par le patients

\begin{figure}[!h]
\begin{center}
\includegraphics[height=8cm,width=18cm]{med.png}
\end{center}
\caption{Dossier medical du patient}
\end{figure}

\subsection{Page d'inscription}

\begin{figure}[!h]
\begin{center}
\includegraphics[height=8cm,width=18cm]{reg.png}
\end{center}
\caption{Page d'inscription}
\end{figure}

\newpage

\subsection{Page de recherche}

\begin{figure}[!h]
\begin{center}
\includegraphics[height=8cm,width=18cm]{search.jpeg}
\end{center}
\caption{Page de recherche}
\end{figure}


\subsection{Interface d'ajout des patients pour l'admin}

\newpage

\begin{figure}[!h]
\begin{center}
\includegraphics[height=6cm,width=18cm]{backadmin.jpeg}
\end{center}
\caption{ajout des patients pour l'admin}
\end{figure}




\section{Conclusion}
La terminaison de cette étape marque la fin de notre projet, il reste à voir le résultat réél de ce projet et en tirer des conclusions, et puis voir des améliorations possible en guise de perspective.



\chapter*{Conclusion générale et perspectives}

Ce stage a été une belle occasion pour travailler en groupe, au
sein d’une entreprise pour développer mes connaissances et
mes compétences et surtout pour travailler avec de nouvelles
technologies



%\chapter*{Bibliographie}

ici c'est la bibliographie






%\input{./existant.tex}

 %\input{./besoins.tex}


%\input{./resultats.tex}



\newpage

%récupérer les citation avec "/footnotemark"
\nocite{*}

%choix du style de la biblio
\bibliographystyle{plain}
%inclusion de la biblio
\bibliography{bibliographie}
%voir wiki pour plus d'information sur la syntaxe des entrées d'une bibliographie
%\thispagestyle{empty}

\end{document}