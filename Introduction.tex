
\chapter*{Introduction générale}



La notion de « serious game », traduite officiellement par la commission générale de
terminologie et de néologie par « jeu sérieux », peut apparaître comme un oxymore. Pourtant
depuis dix ans, il ne s'agit plus d'un simple concept ou d'un projet vague comme celui de « web
sémantique », mais bien d'une réalité suffisamment tangible pour permettre l'existence d'un vrai
marché, avec des producteurs privés et des clients, ou d'un véritable intérêt de la part de
l’Éducation Nationale. Cette dynamique est bien entendu l’un des derniers fruits de la vague
extraordinaire des jeux vidéo depuis quarante ans, ainsi que de l’émergence de la génération des
digital natives ; mais le succès du serious game est tel qu’il a bel et bien débordé largement ces
deux cadres incubateurs. Beaucoup s’y intéressent, parce qu’il est en plein essor et parce qu’il est
efficace ; mieux : il est en plein essor parce qu’il est efficace.
S'il existe plusieurs définitions du serious game, celle du spécialiste français Julian Alvarez,
pionnier de leur étude, fait autorité : « application informatique, dont l'objectif est de combiner à
la fois des aspects sérieux (Serious) tels, de manière non exhaustive, l'enseignement,
l'apprentissage, la communication, ou encore l'information, avec des ressorts ludiques issus du
jeu-vidéo (Game). Une telle association a donc pour but de s'écarter du simple divertissement. »
Il s'agit ainsi, pour simplifier à grands traits, d'un jeu vidéo, le plus souvent en ligne, utilisé à des
fins d'information ou de formation ; ou, de manière ramassée, d'une combinaison sérieuse et
vidéoludique, la seconde étant au service de la première. La vocation d’un serious game est donc
de rendre attrayante la dimension sérieuse par une forme, une interaction, des règles et
éventuellement des objectifs ludiques parmi lesquels, selon le sociologue Roger Caillois, le «goût
pour la difficulté gratuite». Le serious game intéresse potentiellement les entreprises privées, les
associations, le secteur public dont font partie les bibliothèques.
L’entreprise NowEdge, crée en 2020, propose des solutions innovantes de gamification de
l’apprentissage sous forme de serious games destinées aux entreprises et tout établissement offrant
des formations professionnelles.
Ainsi, mon présent travail s’inscrit dans le cadre du stage technique de fin de la
première année en filière GL à l'Ecole nationale supérieure d'informaitque et d'analyse des systèmes pour
l’année universitaire 2022/2023.
