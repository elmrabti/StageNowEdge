
\chapter*{Introduction générale}



La digitalisation de la santé est un processus lancé il y a quelques années à présent et qui commence seulement à prendre des formes concrètes dans la vie des patients, des médecins, des assureurs, des mutuelles, des organismes gouvernementaux et des laboratoires de santé surtout après la pandémie de la covide 19. Cet élan de digitalisation des entreprises s'inscrit dans un schéma global qui touche tous les secteurs de l'économie.

Cette digitalisation passe toujours par les données des patients appartenants au système de santé en questions, ce qui pose quelques problèmes lorsqu'on essaye de suivre les dossiers médicaux des patients sachant que leurs données sont stockées sur plusieurs bases de données, d'où vient l'intérêt de notre projet qui vise à centraliser ce traitement de gestion des données des patients, afin d'offrir aux proffessionnels de santé des interfaces unifiés sur les dossiers de leurs patients.

