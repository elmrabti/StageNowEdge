\chapter{Présentation de l'entreprise}

\section{Présentation de NowEdge: Raisons d'être}

Créée en Avril 2020, NowEdge est une startup créée pour répondre aux trois pains points
qui impactent l’apprentissage et la formation :
\begin{itemize}

\item Le faible taux de mémorisation dans la formation (<50\%):

Le format classique des cours et des formations est caractérisé par un faible taux de
mémorisation des participations.

\item Le format classique ne permet pas de faire de la pratique (immersion, mise en contexte, visualisation réaliste):
En particulier dans des domaines
complexes qui nécessitent de la simulation, des grandes quantités de données et un
environnement virtuel pour pouvoir pratiquer.

\item La formation en présentielle impose des contraintes limitatives:
Organisation physique (salle, formateur…) Nombre de participants limité, format
synchrone et accessibilité limitée.

\end{itemize}

L’entreprise propose des solutions de serious games sous la forme SAAS. Ces serious games
peuvent être utilisés pour les raisons de formation, team building, recrutement, et onboarding.\\

\begin{figure}[!h]
\begin{center}
\includegraphics[height=4cm]{Images/nowedge.png}
\end{center}
\caption{Logo de Nowedge}
\end{figure}

\section{L'equipe}

Les trois fondateurs Monsieur JBEL Youssef (CEO \& Product Owner), Monsieur AHROUM Rida (CTO \& COO), et Madame BENCHEIKH Amina (Responsable
pédagogique) comptent sur leurs riches expériences techniques et en matière de
formation pour mettre en place des solutions toujours plus innovantes sous forme des 5
jeux sérieux. En plus des trois fondateurs, l’équipe est composée également de quatre
freelances qui s’occupent de la partie développement front-end et marketing digital.
 
\section{L'entreprise d'aujourd'hui}

NowEdge crée des Business Games inspirés de la qualité et du contenu des grandes
universités dans le monde ( Harvard, HEC..) à destination d’écoles marocaines souveraines, elle
accompagne les cabinets de recrutement, les RH dans la détection de talents propices aux
besoins humains internes et au développement de la culture et des valeurs de l’entreprise, elle
accompagne de plus les entreprises à accroitre le capital intellect au sein de leurs organisations à
travers des sessions de formations engageantes userfriendly et solides en contenu, et elle innove
en matière de formats pour une expérience créant davantage d’intérêt auprès des apprenants.



\section{La vision des foundateurs}

La vision de NowEdge c’est de devenir le concepteur (et non éditeur) 100\% Made in Morocco de Business \& Serious Games pointus, dédiés à l’enseignement supérieur, au
recrutement et à la formation engageante en entreprise au Maroc et à l'international.
 

\section{Les sulutions de NowEdge}

NowEdge propose une plateforme SaaS de Serious Games engageants pour favoriser
l'expérience collaborateur et renforcer l'intelligence collective dans l'entreprise grâce au
"Learning By Doing". Ces expériences d'apprentissage sont prêtes à l'emploi et destinées à la
formation, au recrutement, à l'intégration et à la consolidation d'équipe dans les entreprises, les
écoles, les cabinets de formation et de recrutement.

\subsection{StratEdge : une guerre stratégique et commerciale}

StratEdge est un jeu sérieux qui permet la mise en situation d’une guerre commerciale et
stratégique entre plusieurs entreprises où l’objectif est de pousser le participant à faire preuve
d’intelligence stratégique afin d'être leader de son marché.

Le jeu est composé de 14 à 20 participants or la durée d'une session peut se dérouler en 3
à 4 heures.

Les notions clés : diagnostic stratégique, processus stratégique d’une entreprise,
analyse SWOT, théorie des jeux, EBITDA, Loi de l’offre et de la demande, équilibre de
Nash.

\subsection{FinEdge : le trading challenge}

FinEdge est un jeu sérieux qui permet la mise en situation réelle du participant dans
un rôle de trader/gérant de portefeuille avec pour objectif d’augmenter la valeur du
portefeuille alloué, avec un nombre illimité de participants.
Le challenge peut durer d’un à six mois, or l’apprenant peut bénéficier d’une
maîtrise des instruments financiers et du fonctionnement du marché, découverte des
stratégies d'investissement et de gestion des risques de marchés

\subsection{LMS Gamifié}

Le LMS Gamifié permet une gestion libre de contenu de ses programmes de
formation pour un maximum d’engagement. Il a un parcours constitué de modules,
notions, quizz, évaluations, serious games, dashboard pour le participant lui permettant
de suivre sa progression, ses points d’expériences, son classement par rapport à la
communauté. Le LMS permet un accès au catalogue de serious games sur différentes
thématiques, permet de faire des diférents quizz pour tester les connaissances et les
acquis des participants, et offre aussi un forum pour échanges entre participants et
formateurs.





